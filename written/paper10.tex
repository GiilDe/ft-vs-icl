\documentclass[runningheads]{llncs}
\usepackage{subfig}
\usepackage{hyperref}
\usepackage{url}

\usepackage{times}
\usepackage{epsfig}
\usepackage{graphicx}
\usepackage{caption}
\usepackage{amsmath}
\usepackage{amssymb}
\usepackage{algorithm}
\usepackage{algpseudocode}
\usepackage{booktabs}
\usepackage{multirow}
\usepackage{bm}
\usepackage{bbm}
\usepackage[capitalise]{cleveref}
\usepackage{arydshln}
\usepackage{wrapfig}

\usepackage{xcolor}
\renewcommand\UrlFont{\color{blue}\rmfamily}
\newcommand\sizeof[1]{\left|#1\right|}


\begin{document}
%
\title{ICL-VS-FT} 
%\thanks{
%This research was supported by the Ministry of Science \& Technology, Israel.}} 

\author{Tomer Bar Natan\inst{1}}

\authorrunning{T.  Bar Natan et al.}
\titlerunning{ICL-VS-FT}

\institute{
Tel-Aviv University, Tel-Aviv, Israel \\
\email{\{tomerb5@mail\}.tau.ac.il}}
\maketitle      

\begin{abstract}
  Large pretrained language models have shown
  surprising in-context learning (ICL)
\keywords{NLP \and LLM \and   ICL \and Linearization }
\end{abstract}
\section{Literature Review}
In-context learning (ICL) is a machine learning approach where a model fine-tunes its knowledge and adapts its behavior based on specific contextual information or examples, allowing it to perform better on tasks related to that context.
It enables models to leverage domain-specific or task-specific knowledge without extensive retraining, making them more versatile and adaptable.
In their work, \cite{NEURIPS2020_1457c0d6} explores the remarkable ability of language models, particularly GPT-3, to learn and perform tasks with minimal examples, demonstrating their potential as versatile few-shot learners.
The authors showcase the models' impressive performance across a wide range of tasks and emphasize their capacity to generalize from limited data, highlighting the transformative impact of these models on various natural language processing applications.

In recent research, there has been a growing interest in understanding the relationship between two key concepts: in-context learning (ICL) and gradient descent (GD)-based fine-tuning, particularly in the context of transformer models (\cite{pmlr-v202-von-oswald23a,2022arXiv221210559D}).
This research seeks to uncover how ICL, which involves adapting and learning in specific contexts, can be effectively integrated with the iterative optimization process of GD, especially when fine-tuning transformer models.
However, the majority of the examination was on models that had relaxed constraints and featured linear attention mechanisms:
\begin{equation}
  LinearAttn(K,V,q)=KV^q
\end{equation}

The paper \cite{pmlr-v202-von-oswald23a}, develops an explicit weight values for a linear self-attention layer, achieving an update equivalent to a single iteration of gradient descent (GD) aimed at minimizing mean squared error. Moreover, the authors demonstrate how multiple self-attention layers can progressively execute curvature adjustments, leading to enhancements over standard gradient descent.
They proposed the following:

Given a 1-head linear attention layer and
the tokens $e_{j} = (x_{j},y_{j})$, for $j = 1, . . . , N$, one can construct key, query and value matrices $W_{K}$, $W_{Q}$, $W_{V}$ as well
as the projection matrix P such that a Transformer step on 
every token $e_j$ is identical to the gradient-induced dynamics $e_j \rightarrow (x_j , y_j ) + (0, - \delta W x_j ) = (x_j , y_j ) + PVK^{T}q_j$
such that $e_j = (x_j , y_j - \delta y_j )$. For the test data token
$(x_{N+1}, y_{N+1})$ the dynamics are identical.

By doing so, they demonstrate the capability of linear attention to execute gradient descent on the deep representations constructed by the transformer.


Another paper (\cite{2022arXiv221210559D}) expand the findings from linear attention to conventional attention mechanisms, substantiating their claims with empirical data.
Inspired by \cite{Aizerman2019TheoreticalFO} and \cite{unknown}, the idea in this is paper to explain language models as meta-optimizers.

Consider $W_0$ and $\Delta W$, both belonging to $\mathbb{R}^{d_{out} \times d_{in}}$, where $W_0$ represents the initial parameter matrix, and $\Delta W$ signifies the updating matrix. Additionally, let $x$ be a member of $\mathbb{R}^{d_{in}}$, serving as the input representation. A linear layer, subject to optimization via gradient descent, can be articulated as follows:
\begin{equation}
  \mathcal{F}(x) = (W_0 + \Delta W)x \label{eq:2}
\end{equation}
In the context of the back-propagation algorithm, the determination of $\Delta W$ entails the aggregation of outer products derived from historical input representations $x'_i \in \mathbb{R}^{d_{in}}$ and their corresponding error signals $e_i \in \mathbb{R}^{d_{out}}$:
\begin{equation}
  \Delta W = \sum_{i} e_i \otimes x'_i \label{eq:3}
\end{equation}
Notably, $e_i$ is the result of scaling historical output gradients by $-\gamma$, the negative learning rate.

By equations \eqref{eq:2} and \eqref{eq:3}, we can derive the dual manifestation of linear layers, optimized through gradient descent, as follows:
\begin{align}
  \begin{split}
    \mathcal{F}(x) &= (W_0 + \Delta W)x \\
    &= W_0x + \Delta Wx \\
    &= W_0x + \sum_{i} (e_i \otimes x'_i)x \\
    &= W_0x + \sum_{i} e_i(x^{'T}_ix) \\
    &= W_0x + \text{LinearAttn}(E, X', x)
  \end{split}
  \label{eq:4}
\end{align}
Here, $E$ denotes historical output error signal values, $X'$ corresponds to historical inputs employed as keys, and $x$ serves as the current input, operating as the query.

Their experiments convincingly reveal that a model fine-tuned through gradient steps and a model prompted with in-context examples appear to perform analogous functions, exhibiting similar behaviors on inputs. 
Additionally, they observe significant similarities in the internal behaviors of these two models.


\section{Background and Preliminaries}
\subsection{Dual Form Between Attention and Linear Layers Optimized by Gradient Descent}

The view of language models as meta-optimizers originates from the presentation of the dual and primal forms of the perceptron \cite{Aizerman2019TheoreticalFO}.
This notion was later expressed in terms key/value/query-attention operation by by \cite{irie22dual,dai2023gpt,pmlr-v202-von-oswald23a} which apply it apply it in the modern context of deep neural networks.
They show that linear layers optimized by gradient descent have a dual representation as linear attention.

Let $W \in \mathbb{R}^{d_{\text{out}} \times d_{\text{in}}}$ be the weight matrix of a linear layer initialized at $W_0$, and let $\mathbf{x}, \mathbf{x}_1, \dots, \mathbf{x}_n  \in \mathbb{R}^{d_{\text{in}}}$ be the input and training examples representation respectively.
One step of gradient descent on the loss function $\mathcal{L}$ with learning rate $\eta$ yields the weight update $\Delta W$.
This update can be written as the outer products of the training examples $\mathbf{x}_1, \dots, \mathbf{x}_n$ and the gradients of their corresponding outputs $\mathbf{e}_i = -\eta \nabla_{W_0 x_i}\mathcal{L}$
\begin{equation}
    \Delta W = \sum_i \mathbf{e}_i \otimes \mathbf{x}^{\prime T}_i.
    \label{equ:dual_comp_2}
\end{equation}

Thus the computation of the optimized linear layer can be formulated as 

\begin{equation}
    \begin{aligned}
        \mathcal{F}(\mathbf{x}) = & \left( W_{0} + \Delta W \right) \mathbf{x} \\
        = & W_{0} \mathbf{x} + \Delta W \mathbf{x} \\
        = & W_{0} \mathbf{x} + \sum_i \left( \mathbf{e}_i \otimes \mathbf{x}_i\right) \mathbf{x} \\
        = & W_{0} \mathbf{x} + \sum_i \mathbf{e}_i \left( \mathbf{x}^{T}_i \mathbf{x} \right) \\
        = & W_{0} \mathbf{x} + \operatorname{LinearAttn} \left( E, X, \mathbf{x} \right), 
    \end{aligned}
    \label{equ:sgd_attn_dual}
\end{equation}
where $\operatorname{LinearAttn}(V, K, \mathbf{q})$ denotes the linear attention operation.
From the perspective of attention we regard training examples $X$ as keys, their corresponding gradients as values, and the current input $\mathbf{x}$ as the query.


\subsection{Understanding Transformer Attention as Meta-Optimization}
\label{sec:icl_dual}
In this section we explain the simplified mathematical view of in-context learning as a process of meta-optimization presented in \cite{dai2023gpt}.
For the purpose of analysis, it is useful to view the change to the output induced by attention to the demonstration tokens as equivalent parameter update $\Delta W_{\text{ICL}}$ that take effect on the original attention parameters.

Let $\mathbf{x} \in \mathbb{R}^{d}$ be the input representation of a query token $t$, and $\mathbf{q} = W_{Q} \mathbf{x} \in \mathbb{R}^{d^{\prime}}$ be the attention query vector. 
We use the relaxed linear attention model, whereby the softmax operation and the scaling factor are omitted:
\begin{equation}
    \begin{aligned}
    \mathcal{F}_{\text{ICL}}(\mathbf{q}) & = \operatorname{LinearAttn}(V, K, \mathbf{q}) \\
    & = W_{V} [X^{\prime}; X] \left( W_{K} [X^{\prime}; X] \right)^T \mathbf{q} \\
    \end{aligned}
    \label{equ:icl_attn}
\end{equation}
where $W_{Q}, W_{K}, W_{V} \in \mathbb{R}^{d^{\prime} \times d}$ are the projection matrices for computing the attention queries, keys, and values, respectively; 
$X$ denotes the input representations of query tokens before $t$; 
$X^{\prime}$ denotes the input representations of the demonstration tokens; 
and $[X^{\prime}; X]$ denotes the matrix concatenation. 


They think of $W_{\text{ZSL}} = W_{V} X \left( W_{K} X \right)^T$ as the initial parameters of a linear layer that is updated by attention to in-context demonstrations.
To see this, note that $W_{\text{ZSL}}$ is the attention result in the zero-shot learning setting where no demonstrations are given (Equation \ref{equ:icl_attn}). 
Following the reverse direction of Equation (\ref{equ:sgd_attn_dual}), you arrive at the dual form of the Transformer attention: 
\begin{equation}
    \begin{aligned}
        \mathcal{F}_{\text{ICL}}(\mathbf{q})
        = & W_{\text{ZSL}} \mathbf{q} + \operatorname{LinearAttn} \left( W_{V} X^{\prime}, W_{K} X^{\prime}, \mathbf{q} \right) \\
        = & W_{\text{ZSL}} \mathbf{q} + \sum_i W_{V} \textbf{x}^{\prime}_i \left( \left( W_{K} \textbf{x}^{\prime}_i \right)^T \mathbf{q} \right) \\
        = & W_{\text{ZSL}} \mathbf{q} + \sum_i \left( W_{V} \textbf{x}^{\prime}_i \otimes \left( W_{K} \textbf{x}^{\prime}_i \right) \right) \mathbf{q} \\
        = & W_{\text{ZSL}} \mathbf{q} + \Delta W_{\text{ICL}} \mathbf{q} \\
        = & \left( W_{\text{ZSL}} + \Delta W_{\text{ICL}} \right) \mathbf{q}. 
    \end{aligned}
    \label{equ:icl_opti_dual}
\end{equation}

By analogy with Equation(\ref{equ:sgd_attn_dual}), we can regard $W_{K} \textbf{x}^{\prime}_i$ as the training examples and $W_{V} X^{\prime}$ as their corresponding meta-gradients. 

\section{Method}
In the following sections we describe the evaluation metrics used to compare the behavior of ICL and finetuning.
To ensure an optimal comparison, we have adopted the identical metrics as introduced in \cite{dai2023gpt}:

\subsection*{Prediction Recall}

From the perspective of model prediction, models with similar behavior should have aligned predictions.
We measure the recall of correct ICL predictions to correct finetuning predictions.
Given a set of test examples, we count the subsets of examples correctly predicted by each model: $C_{\text{ZSL}}, C_{\text{ICL}}, C_{\text{FT}}$.
To compare the update each method induces to the model's prediction we subtract correct predictions made in the ZSL setting.
Finally we compute the \textbf{Rec2FTP} score as: $\frac{ \sizeof{ \left( C_{\text{ICL}} \cap C_{\text{FT}} \right) \setminus C_{\text{ZSL}} } }{ \sizeof{ C_{\text{FT}} \setminus C_{\text{ZSL}} } }$ .
A higher Rec2FTP score suggests that ICL covers more correct behavior of finetuning from the perspective of the model prediction.

%This measure is agnostic to the inner workings of the attention mechanism.

\subsection*{Attention Output Direction}
In the context of an attention layer's hidden state representation space within a model, we analyze the modifications made to the attention output representation (\textbf{SimAOU}).

For a given query example, let $h^{(l)}_X$ represent the normalized output representation of the last token at the $l$-th attention layer within setting $X$. The alterations introduced by ICL and finetuning in comparison to ZSL are denoted as $h^{(l)}_{ICL} - h^{(l)}_{ZSL}$ and $h^{(l)}{FT} - h^{(l)}_{ZSL}$, respectively. We calculate the cosine similarity between these two modifications to obtain SimAOU ($\Delta FT$) at the $l$-th layer. A higher SimAOU ($\Delta FT$) indicates that ICL is more inclined to adjust the attention output in the same direction as finetuning.
For the sake of comparison, we also compute a baseline metric known as SimAOU (Random $\Delta$), which measures the similarity between ICL updates and updates generated randomly.

\subsection*{Attention Map Similarity}
We use SimAM to measure the similarity between attention maps and query tokens for ICL and finetuning.
For a query example, let $m^{(l,h)}_X$ represent the attention weights before softmax in the $h$-th head of the $l$-th layer for setting $X$. In ICL, we focus solely on query token attention weights, excluding demonstration tokens. Initially, before finetuning, we compute the cosine similarity between $m^{(l,h)}_{ICL}$ and $m^{(l,h)}_{ZSL}$, averaging it across attention heads to obtain SimAM (Before Finetuning) for each layer.
Similarly, after finetuning, we calculate the cosine similarity between $m^{(l,h)}_{ICL}$ and $m^{(l,h)}_{FT}$ to obtain SimAM (After FT). A higher SimAM (After FT) relative to SimAM (Before FT) indicates that ICL's attention behavior aligns more with a finetuned model than a non-finetuned one.

 \bibliographystyle{splncs04}
 \bibliography{refs}




\end{document}


